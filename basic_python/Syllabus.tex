\documentclass{article}
\usepackage{fancyhdr}
\usepackage{listings}
\usepackage{geometry}
\usepackage{hyperref}

\geometry{letterpaper, left=2cm, right=2cm, top=2cm, bottom=2cm}

\pagestyle{fancy}
\fancyhf{}
\renewcommand{\headrulewidth}{0pt}
\fancyfoot[LE,RO]{\thepage}
\fancyfoot[RE,LO]{\copyright\hspace{0.25em}codeabode 2025. \href{https://codeabode.co}{\underline{codeabode.co}}}

\title{\vspace{-3em}Python Ping-Pong Syllabus\vspace{-3em}}

\begin{document}
\fontsize{14}{16}\selectfont
\maketitle
\tableofcontents

\section{Introduction}
This course is derived from the \href{https://cs50.harvard.edu/python/2022/syllabus/}{\underline{The Harvard CS50 Syllabus}} and study materials are derived from AP Computer Science A. In addition to the skills learned in that course, our Python syllabus prioritizes Python expertise for \href{https://www.pygame.org/wiki/about}{Pygame Development}. \\

It is highly recommended that you review fractions with your kid as we move along the course.

\section{Basic Variables and Print}
The student will spend this first class understanding how to create strings and integers and print to the console. \\

\begin{itemize}
    \item Strings
    \item Integers
    \item Floats (print output trimming)
\end{itemize}

\begin{lstlisting}[language=Python]
print("Hello, World!")
print(9)
\end{lstlisting}

The student will learn to use variables to store data and print multiple variables along with a sentence.

\begin{lstlisting}[language=Python]
a = 5
b = 10
print("a is:", a, "and b is:", b)
\end{lstlisting}

No floats or booleans yet to avoid confusion. Keep it simple.

Teach operators:
\begin{itemize}
    \item \verb|+|
    \item \verb|-|
    \item \verb|*|
    \item \verb|**|
    \item \verb|/| (fallacies of integer division)
    \item \verb|%|
\end{itemize}

\section{If Statements}

\begin{itemize}
    \item \verb|if|, \verb|elif|, and \verb|else|
    \item \verb|and|, \verb|or|, and \verb|not|
    \item \verb|<|, \verb|<=|, \verb|>|, \verb|>=|, \verb|==|, and \verb|!=|
    \item Booleans data type and storing conditions in a variable
\end{itemize}

At this point you can start debugging assignments for certain days and make them a little bit harder too.

Use truth tables to teach advanced concepts like DeMorgan's law.

For practice, use the following repos:


\begin{itemize}
    \item \href{https://github.com/codeabode101/vars-print-basic-if-practice}{Basic Practice without input}
    \item \href{https://github.com/codeabode101/program-flow-practice}{Program Flow Practice with input}
    \item \href{https://github.com/codeabode101/basic-types-advanced-handling-practice}{Advanced Handling (debug skills development)}
    \item \href{https://github.com/codeabode101/critical-thinking-if-vars-practice}{Multiple Choice type critical thinking problem set}
    \item \href{https://runestone.academy/ns/books/published/csawesome/Unit3-If-Statements/Exercises.html}{ Difficult multiple choice/debug problems}
\end{itemize}


\subsection{The Boolean Labyrinth}
\begin{itemize}
    \item The String Gate: Enter a password that must contain A and end with Z, or be '42'
    \item The Integer Trap: Enter a number that's even, not divisible by 5, and > 10
    \item The Boolean Paradox: Enter either True or False. A cat is NOT a dog AND 5 is a string
\end{itemize}

\section{Calculator Project}

Make a simple calculator that adds, subtracts, multiplies, and divides.
You can \verb|input()| the number, symbol and another number. At the end, the entire equation should be printed. (Example: \verb|1+1=2|) \\

This project tests type casting and basic if usage, which are the more difficult concepts. \\

At this point it is important to determine what skills need solidification before moving on. 

\section{String Methods}

\begin{itemize}
    \item \verb|upper()|
    \item \verb|lower()|
    \item \verb|strip()|
    \item \verb|replace()|
    \item \verb|split()|
\end{itemize}

Go back and use \verb|split()| for the calculator project.


\section{Python Turtle}

Use graphics to draw simple shapes. Use if statements for complex logic.

\begin{itemize}
    \item \verb|turtle|
    \item \verb|goto()|
    \item \verb|forward()|
    \item \verb|left()|
    \item \verb|right()|
    \item \verb|done()|
\end{itemize}

Sample assignments:

\begin{itemize}
    \item Zig-Zag line
    \item Traffic Light from input
\end{itemize}

\section{Loops}

\begin{itemize}
    \item \verb|for| and \verb|while|
    \item \verb|in|
    \item \verb|break| and \verb|continue|
    \item \verb|range()|
    \item \verb|len()|
    \item \verb|enumerate()| 
\end{itemize}

Also at this point you want to learn:

\begin{itemize}
    \item \verb|match|
    \item Maybe: \verb|.sort()|
\end{itemize}

Now begin to draw advanced shapes using Turtle.

\begin{itemize}
    \item Shape size challenge
    \item Turtle Race Game where you count steps in a loop
\end{itemize}

\section{Debugging}
As loops can get buggy, it is important to start debugging now.

\begin{itemize}
    \item \verb|assert|
    \item \verb|print|
    \item comments for advanced control flow
    \item VSCode debugger
\end{itemize} 

\section{Advanced Datatypes}

\begin{itemize}
    \item Lists
    \item Dictionaries (store advanced data)
\end{itemize}


Potentially may return to teach the following concepts although not immediately necessary:

\begin{itemize}
    \item Tuples
    \item Sets (no duplicates)
\end{itemize}


Review the old concepts but up the difficulty of assignments. 

% Add to "Debugging"
\subsection{Unit Testing for Games}
\begin{itemize}
    \item Write tests for game mechanics (e.g., \verb|assert player.jump() == True|)
    \item Use \verb|pytest| for automated checks
\end{itemize}

\section{Functions}

\begin{itemize}
    \item \verb|def|
    \item \verb|return|
    \item Return Types
    \item Call Stack Basics
    \item Global variables
\end{itemize}

\section{Classes}

\begin{itemize}
    \item \verb|class|
    \item Objects
    \item Methods
    \item Inheritance
    \item Initializing a Class
\end{itemize}

\section{Libraries}

\begin{itemize}
    \item \verb|import|
    \item \verb|from|
    \item \verb|random|
    \item \verb|time.sleep()|
\end{itemize}
% Add to "Libraries" section
\subsection{CSV and Data Files}
\begin{itemize}
    \item Save player stats with \verb|csv.writer()|
    \item Load high scores with \verb|csv.reader()|
\end{itemize}

\section{Exception}

\begin{itemize}
    \item \verb|try|, \verb|except|, and \verb|raise|
\end{itemize}

\section{File I/O}

\begin{itemize}
    \item \verb|open()|
    \item \verb|read()|
    \item \verb|write()|
    \item \verb|close()|
\end{itemize}

At this point, we will start the flagship "code that writes code" assignment.

\section{sys Module}

Introduction to Linux and basic concepts.

\begin{itemize}
    \item \verb|sys.exit()|
    \item \verb|sys.argv|
    \item How to run other Python files
\end{itemize}

Additional libraries like\verb|math| may be taught, but aren't necessary for PyGame.

\section{Intro to Pygame}
???

\subsection{Weather App with Pygame}
\begin{itemize}
    \item HTTP requests
    \item JSON handling
    \item Rendering
\end{itemize}

% New section
\section{Advanced Pygame Techniques}
\begin{itemize}
    \item Sprite animation with \verb|pygame.image.load()|
    \item Collision detection using rectangles
    \item Playing sound effects and music
\end{itemize}

% upload to https://itch.io/games/made-with-pygame
\end{document}
