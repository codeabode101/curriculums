\documentclass{article}
\usepackage{fancyhdr}
\usepackage{listings}
\usepackage{geometry}
\usepackage{hyperref}

\geometry{letterpaper, left=2cm, right=2cm, top=2cm, bottom=2cm}

\pagestyle{fancy}
\fancyhf{}
\renewcommand{\headrulewidth}{0pt}
\fancyfoot[LE,RO]{\thepage}
\fancyfoot[RE,LO]{\copyright\hspace{0.25em}codeabode 2025. \href{https://codeabode.co}{\underline{codeabode.co}}}

\title{\vspace{-3em}Python Ping-Pong Syllabus\vspace{-3em}}

\begin{document}
\fontsize{14}{16}\selectfont
\maketitle
\tableofcontents

\section{Introduction}
This course is derived from the \href{https://cs50.harvard.edu/python/2022/syllabus/}{\underline{The Harvard CS50 Syllabus}} and study materials are derived from AP Computer Science A. In addition to the skills learned in that course, our Python syllabus prioritizes Python expertise for \href{https://www.pygame.org/wiki/about}{Pygame Development}. \\

It is highly recommended that you review fractions with your kid as we move along the course.

\section{Basic Variables and Print}
The student will spend this first class understanding how to create strings and integers and print to the console. \\

\begin{lstlisting}[language=Python]
print("Hello, World!")
print(9)
\end{lstlisting}

The student will learn to use variables to store data and print multiple variables along with a sentence.

\begin{lstlisting}[language=Python]
a = 5
b = 10
print("a is:", a, "and b is:", b)
\end{lstlisting}

No floats or booleans yet to avoid confusion. Keep it simple.

\section{Basic If Statements}

\begin{itemize}
    \item \verb|if|, \verb|elif|, and \verb|else|
    \item \verb|and|, \verb|or|, and \verb|not|
    \item \verb|<|, \verb|<=|, \verb|>|, \verb|>=|, \verb|==|, and \verb|!=|
\end{itemize}

At this point you can start debugging assignments for certain days and make them a little bit harder too.

\section{Advanced Datatypes}

\begin{itemize}
    \item Booleans (connect with if)
    \item Floats (print output trimming)
    \item Lists
    \item Dictionaries (store advanced data)
\end{itemize}

Potentially may return to teach the following concepts although not immediately necessary:

\begin{itemize}
    \item Tuples
    \item Sets (no duplicates)
\end{itemize}

\section{Loops}

\begin{itemize}
    \item \verb|for| and \verb|while|
    \item \verb|in|
    \item \verb|break| and \verb|continue|
    \item \verb|range()|
    \item \verb|len()|
    \item \verb|enumerate()| 
\end{itemize}

Also at this point you want to learn:

\begin{itemize}
    \item \verb|match|
    \item Maybe: \verb|.sort()|
\end{itemize}

\section{Debugging}
As loops can get buggy, it is important to start debugging now.

\begin{itemize}
    \item \verb|assert|
    \item \verb|print|
    \item comments for advanced control flow
    \item VSCode debugger
\end{itemize} 

Review the old concepts but up the difficulty of assignments. 

% Add to "Debugging"
\subsection{Unit Testing for Games}
\begin{itemize}
    \item Write tests for game mechanics (e.g., \verb|assert player.jump() == True|)
    \item Use \verb|pytest| for automated checks
\end{itemize}

\section{Functions}

\begin{itemize}
    \item \verb|def|
    \item \verb|return|
    \item Return Types
    \item Call Stack Basics
    \item Global variables
\end{itemize}

\section{Classes}

\begin{itemize}
    \item \verb|class|
    \item Objects
    \item Methods
    \item Inheritance
    \item Initializing a Class
\end{itemize}

\section{Libraries}

\begin{itemize}
    \item \verb|import|
    \item \verb|from|
    \item \verb|random|
    \item \verb|time.sleep()|
\end{itemize}
% Add to "Libraries" section
\subsection{CSV and Data Files}
\begin{itemize}
    \item Save player stats with \verb|csv.writer()|
    \item Load high scores with \verb|csv.reader()|
\end{itemize}

\section{Exception}

\begin{itemize}
    \item \verb|try|, \verb|except|, and \verb|raise|
\end{itemize}

\section{File I/O}

\begin{itemize}
    \item \verb|open()|
    \item \verb|read()|
    \item \verb|write()|
    \item \verb|close()|
\end{itemize}

At this point, we will start the flagship "code that writes code" assignment.

\section{sys Module}

Introduction to Linux and basic concepts.

\begin{itemize}
    \item \verb|sys.exit()|
    \item \verb|sys.argv|
    \item How to run other Python files
\end{itemize}

Additional libraries like\verb|math| may be taught, but aren't necessary for PyGame.

\section{Intro to Pygame}
???

\subsection{Weather App with Pygame}
\begin{itemize}
    \item HTTP requests
    \item JSON handling
    \item Rendering
\end{itemize}

% New section
\section{Advanced Pygame Techniques}
\begin{itemize}
    \item Sprite animation with \verb|pygame.image.load()|
    \item Collision detection using rectangles
    \item Playing sound effects and music
\end{itemize}
\end{document}
