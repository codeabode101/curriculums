\documentclass{article}
\usepackage{fancyhdr}
\usepackage{geometry}
\usepackage{hyperref}

\geometry{letterpaper, left=2cm, right=2cm, top=2cm, bottom=2cm}

\pagestyle{fancy}
\fancyhf{}
\renewcommand{\headrulewidth}{0pt}
\fancyfoot[LE,RO]{\thepage}
\fancyfoot[RE,LO]{\copyright\hspace{0.25em}codeabode 2025. \href{https://codeabode.co}{\underline{codeabode.co}}}

\title{\vspace{-3em}Assignment: Print \& Basic Vars\vspace{-3em}}

\begin{document}
\fontsize{14}{16}\selectfont
\maketitle
\vspace{-1em}
\section{In Class}
The student learned the following:
\begin{itemize}
    \item How to navigate VSCode
    \item How to print to the console (basic)
    \item Basic Python syntax
    \item How to create numerical and string variables
    \item How to print multiple variables at once 
\end{itemize}

It is top priority that these basics solidify the student's understanding of basic Python handling.

\section{Homework}
By "Python profile", this section means holding all information specific to the person in variables, like name, age, or hobbies. Printing should look like  \\ \\
\verb|print(name, " is ", verb)|

\subsection{Sandra's Profile}

Create a Python Profile for the following person. Then print the relevant information.

\begin{center}
    I'm Sandra! I live in Seattle. My favorite color is green. I am 26 years old. I love to play Ice Hockey!
\end{center}

Each of the following should be stored in variables:

\begin{itemize}
    \item Name
    \item Location
    \item Color
    \item Age
    \item Hobby
\end{itemize}

Print each variable to the console in a sentence-like format, making use of comma to split different variables as learned in class.

\subsection{Anisha's Profile}

Create a Python Profile for the following person. Then print the relevant information.

\begin{center}
    I'm Anisha! I go to High School North. In my free time, I like to surf and play video games.
\end{center}

Store the relevant information specific to Anisha in variables. \textbf{Print should only be used for the sentence structure.}

\subsection{Challenge}

Create a Python profile for yourself using variables holding (5) pieces of information like your name, age, or hobbies. Then add 5 years to your age with \verb|age += 5|. Then update the information as you see relevant. \\ 

Print the output in a structured and readable format. 

\section{Time Spent}

Each homework session should be no more than an hour. 30-45 minutes on average is recommended. If you are stuck, feel free to reach out. If an assignment feels redundant, please let your instructor know before skipping. At this stage, \textbf{the more practice, the better.}

\end{document}
