\documentclass{article}
\usepackage{enumitem}
\usepackage{fancyhdr}
\usepackage{geometry}
\usepackage{listings}
\usepackage{multicol}
\usepackage{hyperref}
\usepackage{tasks}

\geometry{letterpaper, left=1cm, right=1cm, top=2cm, bottom=2cm}

\pagestyle{fancy}
\fancyhf{}
\renewcommand{\headrulewidth}{0pt}
\fancyfoot[LE,RO]{\thepage}
\fancyfoot[RE,LO]{\copyright\hspace{0.25em}codeabode 2025. \href{https://codeabode.co}{\underline{codeabode.co}}}

\setlength{\columnsep}{0.5cm}

\title{\vspace{-3em}Types Worksheet\vspace{-2em}}

\begin{document}

\maketitle


    \begin{enumerate}
        \item What does this code output?
        \begin{lstlisting}
i = 1  
while i <= 5:  
    print(str(i) * i)  
    i += 1  
        \end{lstlisting}
        \rule{\linewidth}{0.4pt} % Full-width bar
        \rule{\linewidth}{0.4pt} % Full-width bar
        \rule{\linewidth}{0.4pt} % Full-width bar
        \rule{\linewidth}{0.4pt} % Full-width bar
        \rule{\linewidth}{0.4pt} % Full-width bar
        \rule{\linewidth}{0.4pt} % Full-width bar
        \rule{\linewidth}{0.4pt} % Full-width bar

        \item What does this code output?
        \begin{lstlisting}
k = 0  
while k < 20:  
    if k % 3 == 1:  
        print(k, end=' ')  
    k += 2  
        \end{lstlisting}

        \rule{\linewidth}{0.4pt} % Full-width bar
        \rule{\linewidth}{0.4pt} % Full-width bar
        \rule{\linewidth}{0.4pt} % Full-width bar
        \rule{\linewidth}{0.4pt} % Full-width bar
        \rule{\linewidth}{0.4pt} % Full-width bar
        \rule{\linewidth}{0.4pt} % Full-width bar
        \rule{\linewidth}{0.4pt} % Full-width bar

        \item What does this code output?
        \begin{lstlisting}
i = 5  
while i >= 1:  
    j = 0  
    while j < i:  
        print(i, end='')  
        j += 1  
    print()  
    i -= 1  
        \end{lstlisting}
        \rule{\linewidth}{0.4pt} % Full-width bar
        \rule{\linewidth}{0.4pt} % Full-width bar
        \rule{\linewidth}{0.4pt} % Full-width bar
        \rule{\linewidth}{0.4pt} % Full-width bar
        \rule{\linewidth}{0.4pt} % Full-width bar
        \rule{\linewidth}{0.4pt} % Full-width bar
        \rule{\linewidth}{0.4pt} % Full-width bar

        \item What does this code output?
        \begin{lstlisting}
a = 0  
b = 2  
while b != 0:  
    if a / b < 0.5:  # Short-circuit prevents division by zero  
        print(f"{a}/{b}", end=' ')  
    a += 1  
    b -= 1  
        \end{lstlisting}
        \rule{\linewidth}{0.4pt} % Full-width bar
        \rule{\linewidth}{0.4pt} % Full-width bar
        \rule{\linewidth}{0.4pt} % Full-width bar
        \rule{\linewidth}{0.4pt} % Full-width bar
        \rule{\linewidth}{0.4pt} % Full-width bar
        \rule{\linewidth}{0.4pt} % Full-width bar
        \rule{\linewidth}{0.4pt} % Full-width bar

        \item What does this code output?
        \begin{lstlisting}
rows = 3  
i = 0  
while i < rows:  
    j = 0  
    while j <= i:  
        print(rows - i, end='')  
        j += 1  
    print()  
    i += 1  
        \end{lstlisting}
        \rule{\linewidth}{0.4pt} % Full-width bar
        \rule{\linewidth}{0.4pt} % Full-width bar
        \rule{\linewidth}{0.4pt} % Full-width bar
        \rule{\linewidth}{0.4pt} % Full-width bar
        \rule{\linewidth}{0.4pt} % Full-width bar
        \rule{\linewidth}{0.4pt} % Full-width bar
        \rule{\linewidth}{0.4pt} % Full-width bar


    \item What does this code output?
    \begin{lstlisting}
x = 3
y = 0
while x > 0 and y < 2:
    x -= 1
    y += 1
    print(x, y)
    \end{lstlisting}
    \rule{\linewidth}{0.4pt} % Full-width bar
    \rule{\linewidth}{0.4pt} % Full-width bar
    \rule{\linewidth}{0.4pt} % Full-width bar

    \item What does this code output?
    \begin{lstlisting}
a = 5
b = 3
while a > b:
    print(a, b)
    a -= 2
    b += 1
    if a <= b:
        print("TERMINATE")
    \end{lstlisting}
    \rule{\linewidth}{0.4pt} % Full-width bar
    \rule{\linewidth}{0.4pt} % Full-width bar
    \rule{\linewidth}{0.4pt} % Full-width bar
    \rule{\linewidth}{0.4pt} % Full-width bar

    \item What does this code output?
    \begin{lstlisting}
p = 0
q = 4
while q > 0:
    p = p + (q % 2)
    print(p, end=' ')
    q -= 1
    \end{lstlisting}
    \rule{\linewidth}{0.4pt} % Full-width bar
    \rule{\linewidth}{0.4pt} % Full-width bar
    \rule{\linewidth}{0.4pt} % Full-width bar

    \item What does this code output?
    \begin{lstlisting}
m = 1
n = 5
while n > m:
    if (n + m) % 3 == 0:
        print(n, m)
    n -= 2
    m += 1
    \end{lstlisting}
    \rule{\linewidth}{0.4pt} % Full-width bar
    \rule{\linewidth}{0.4pt} % Full-width bar
    \rule{\linewidth}{0.4pt} % Full-width bar

    \item Explain why the final print statement is unreachable:
    \begin{lstlisting}
r = 4
s = 1
while r > s:
    print(r, s)
    r -= 1
    s += 1
    if r == s:
        print("MIDPOINT")
print("FINAL")  # Why is this unreachable?
    \end{lstlisting}
    \rule{\linewidth}{0.4pt} % Full-width bar
    \rule{\linewidth}{0.4pt} % Full-width bar
    \rule{\linewidth}{0.4pt} % Full-width bar
    \rule{\linewidth}{0.4pt} % Full-width bar
    \item What is printed? Hint: Google "string subscripts in python"
    \begin{lstlisting}
s = "AbCdEf"
i = 0
res = ""
while i < len(s):
    if i % 2 == 0:
        res += s[i].lower()
    else:
        res += s[i].upper()
    i += 1
print(res)
    \end{lstlisting}
    \rule{\linewidth}{0.4pt}
    
    \item What is printed?
    \begin{lstlisting}
text = "HELLO"
i = 0
output = ""
while i < len(text):
    if text[i].lower() in "aeiou":
        output += '*'
        break
    else:
        output += text[i].lower()
    i += 1
print(output)
    \end{lstlisting}
    \rule{\linewidth}{0.4pt}
    
    \item What is printed?
    \begin{lstlisting}
word = "c0d3ab0de"
i = 0
total = 0
while i < len(word):
    if word[i].isdigit():
        total += int(word[i]) * (i if i % 2 == 0 else -i)
    i += 1
print(total)
    \end{lstlisting}
    \rule{\linewidth}{0.4pt}
    
    \item What is printed?
    \begin{lstlisting}
s = "a1b2c3d4"
k = 0
result = 0
while k < len(s):
    if s[k].isalpha():
        result += ord(s[k].lower()) - 96
    else:
        result -= int(s[k])
    k += 1
print(result)
    \end{lstlisting}
    \rule{\linewidth}{0.4pt}
    
    \item What is printed?
    \begin{lstlisting}
text = "NaN"
count = 0
while text.lower() != "infinity":
    print(text, end=" ")
    count += 1
    text = text[1:] + text[0]
    if count >= 5:
        break
    \end{lstlisting}
    \rule{\linewidth}{0.4pt}
    
    \item What is printed?
    \begin{lstlisting}
phrase = "LoOpS"
i = 0
output = ""
while i < len(phrase):
    if phrase[i].isupper():
        output += phrase[i].lower()
    elif phrase[i].islower():
        output += str(i)
    else:
        output += '?'
    i += 1
print(output)
    \end{lstlisting}
    \rule{\linewidth}{0.4pt}
    
    \item What is printed?
    \begin{lstlisting}
s = "x3y2z1"
i = 0
res = ""
while i < len(s):
    if s[i].isalpha():
        char = s[i].lower()
        num = int(s[i+1])
        res += char * num
        i += 1  # Skip next digit
    i += 1
print(res)
    \end{lstlisting}
    \rule{\linewidth}{0.4pt}
    
    \item What is printed?
    \begin{lstlisting}
code = "A1B2C3"
i = 0
sum_val = 0
while i < len(code):
    if i % 2 == 0:  # Even index (0-indexed)
        sum_val += ord(code[i].lower()) - 96
    else:
        sum_val -= int(code[i])
    i += 1
print(sum_val)
    \end{lstlisting}
    \rule{\linewidth}{0.4pt}
    
    \item What is printed?
    \begin{lstlisting}
text = "PYTHON"
i = len(text) - 1
output = ""
while i >= 0:
    if (len(text) - i) % 2 == 1:
        output += text[i].lower()
    else:
        output += text[i]
    i -= 1
print(output)
    \end{lstlisting}
    \rule{\linewidth}{0.4pt}
    
    \item Why does this loop terminate and what is printed?
    \begin{lstlisting}
s = "TRUTH"
count = 0
while "false".upper() != "FALSE" or count < 2:
    print(s[count % len(s)], end="")
    count += 1
    if count > len(s) * 2:
        break
    \end{lstlisting}
    \rule{\linewidth}{0.4pt}
\end{enumerate}
\end{document}

