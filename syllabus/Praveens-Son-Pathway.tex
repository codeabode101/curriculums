\documentclass{article}
\usepackage{fancyhdr}
\usepackage{listings}
\usepackage{geometry}
\usepackage{hyperref}
\usepackage{enumitem}
\usepackage{amsmath}
\usepackage[most]{tcolorbox}

\geometry{letterpaper, left=2cm, right=2cm, top=2cm, bottom=2cm}

\pagestyle{fancy}
\fancyhf{}
\renewcommand{\headrulewidth}{0pt}
\fancyfoot[LE,RO]{\thepage}
\fancyfoot[RE,LO]{\copyright\hspace{0.25em}codeabode 2025. \href{https://codeabode.co}{\underline{codeabode.co}}}

\title{\vspace{-3em}Game Development Pathway for Praveen's Son\vspace{-3em}}
\author{}

\newtcolorbox{learningblock}{colback=blue!5!white, colframe=blue!75!black, 
fonttitle=\bfseries, title=Learning Pathway, arc=3mm, boxsep=2mm}

\begin{document}
\fontsize{14}{16}\selectfont
\maketitle

\section{Personalized Learning Approach}
This course is designed specifically for your son, building on his interest in games to develop practical programming skills. Recognizing the value of hands-on learning, we'll:

\begin{itemize}[leftmargin=*, noitemsep]
    \item Start from his current interests and expand his skills
    \item Use project-based learning with immediate visual feedback
    \item Eliminate theoretical lectures - everything will be hands-on
    \item Develop troubleshooting skills through guided problem-solving
    \item Build confidence through achievable milestones
\end{itemize}

\section{Course Structure \& Methodology}
\begin{learningblock}
\textbf{Weekly Sessions:} 
\begin{itemize}[noitemsep]
\item 60 minute hands-on coding sessions 
\item Flexible scheduling (recommended weekly)
\item Focused on student's questions and project progress
\end{itemize}

\textbf{Homework Philosophy:}
\begin{itemize}[noitemsep]
\item Purpose-driven mini-games (not exercises)
\item Designed for 15-30 minute daily practice
\item Optional challenges for deeper exploration
\item Midweek progress check-ins available
\end{itemize}

\textbf{Learning Foundation:}
\begin{itemize}[noitemsep]
\item Concepts introduced through game mechanics
\item Math taught visually through game physics
\item Just-in-time learning - no boring lectures!
\end{itemize}
\end{learningblock}

\section{Initial Skills Assessment}
In the first session, your son will create a simple game prototype to demonstrate his approach to building games:

\begin{center}
\begin{tabular}{|l|l|}
\hline
\textbf{Game Choice} & \textbf{Learning Focus} \\
\hline
Ping-Pong & Physics, collision detection \\
\hline
Space Invaders & Enemy patterns, scoring \\
\hline
Dinosaur Runner & Obstacle generation \\
\hline
Student's Choice & Creative design \\
\hline
\end{tabular}
\end{center}

\subsection*{What We'll Observe}
\begin{itemize}
\item \textbf{Problem-solving}: Approach to challenges
\item \textbf{Resource usage}: Documentation/Google skills
\item \textbf{Debugging}: Response to errors
\item \textbf{Creativity}: Unique solutions to game mechanics
\end{itemize}

The specific game doesn't matter as much as understanding his approach and interests to personalize the learning journey.

\section{Python Foundations}
\begin{itemize}[nosep]
    \item \textbf{Reference:} \href{https://codeabode.co/Syllabus.pdf}{Core Syllabus} (Variables, Control Flow, etc.)
    \item \textbf{Game Connection:} Basic game mechanics and logic
    \item \textbf{First Project:} Simple interactive story game
    \item \textbf{Pace:} Determined by student's comfort level
\end{itemize}

\section{Core Game Development}
\begin{itemize}[nosep]
    \item Pygame basics: Sprites, collision, animation
    \item Game physics: Movement, gravity, collisions
    \item \textbf{Project:} Custom platformer game
    \item \textbf{Progress:} Move forward when core mechanics work
\end{itemize}

\section{Advanced Game Features}
\begin{itemize}[nosep]
    \item \textbf{Level Design:} Creating engaging challenges
    \begin{itemize}
        \item Obstacle placement
        \item Difficulty progression
        \item Power-up systems
    \end{itemize}
    \item \textbf{Project:} Multi-level adventure game
    \item \textbf{Trigger:} When basic game is functional
\end{itemize}

\section{AI for Game Enhancement}
\begin{itemize}[nosep]
    \item \textbf{Smart Enemies:} Basic AI behaviors
    \begin{itemize}
        \item Pathfinding algorithms
        \item Pattern-based movement
        \item Adaptive difficulty
    \end{itemize}
    \item \textbf{Project:} Strategy game with intelligent opponents
    \item \textbf{Trigger:} When ready for more complex challenges
\end{itemize}

\section{Game Polish \& Publishing}
\begin{itemize}[nosep]
    \item \textbf{Polishing:} Menus, sound effects, visual effects
    \item \textbf{Optimization:} Performance improvements
    \item \textbf{Publishing:} Packaging for sharing
    \item \textbf{Capstone:} Completed game to share with friends
    \item \textbf{Trigger:} When core game is complete
\end{itemize}

\section{Learning Pathway}

\begin{learningblock}
\textbf{Student-Led Progression:}
\begin{itemize}[nosep]
    \item Control over learning speed and game choices
    \item Concepts introduced when relevant to current project
    \item Advancement based on project milestones
\end{itemize}

\textbf{Game Development Pathway:}
\begin{center}
\small
\begin{tabular}{p{3.5cm} p{3cm} p{6cm}}
\textbf{Skill} & \textbf{Timing} & \textbf{Game Application} \\
\hline
Python Basics & First 1-2 sessions & Game logic, scoring systems \\
\hline
Pygame Fundamentals & When ready for visuals & Character control, graphics \\
\hline
Game Physics & First platformer & Movement, collisions, jumps \\
\hline
AI Behaviors & Mid-course & Smart enemies, NPCs \\
\hline
Polish \& Publishing & Final project & Menus, effects, sharing \\
\end{tabular}
\end{center}
\end{learningblock}

\section{Improvements}
If you seek improvements to this syllabus or a change in the general direction, let me know and the necessary modifications will be made.

\end{document}
