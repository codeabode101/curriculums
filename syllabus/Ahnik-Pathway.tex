\documentclass{article}
\usepackage{fancyhdr}
\usepackage{geometry}
\usepackage{enumitem}
\usepackage{hyperref}
\usepackage[most]{tcolorbox}

\geometry{a4paper, left=1.5cm, right=1.5cm, top=1.5cm, bottom=1.5cm}

\pagestyle{fancy}
\fancyhf{}
\renewcommand{\headrulewidth}{0pt}
\fancyfoot[LE,RO]{\thepage}
\fancyfoot[RE,LO]{\copyright\hspace{0.25em}codeabode 2025. \href{https://codeabode.co}{\underline{codeabode.co}}}

\title{\vspace{-2em}Pathway for Ahnik\vspace{-2em}}
\author{}
\date{}

\newtcolorbox{learningblock}{colback=blue!5!white, colframe=blue!75!black, 
fonttitle=\bfseries, title=Learning Pathway, arc=3mm, boxsep=2mm}

\begin{document}
\fontsize{13}{15}\selectfont
\maketitle

\section{Personalized Learning Approach}
This course is designed specifically for Ahnik so that he develops \textbf{creativity} and \textbf{problem solving} through the lens of game development, setting Ahnik for success whatever his interests may be, coding related or not.

We will:

\begin{itemize}[leftmargin=*, noitemsep]
    \item Start from his current interests and past classes to expand his skills
    \item Use project-based learning with immediate visual feedback
    \item Eliminate theoretical lectures - everything will be hands-on
    \item Develop troubleshooting skills through guided problem-solving
    \item Build confidence through achievable milestones
\end{itemize}

\section{Course Structure \& Methodology}
\begin{learningblock}
\textbf{Weekly Sessions:} 
\begin{itemize}[noitemsep]
\item 60 minute hands-on coding sessions 
\item Flexible scheduling (recommended weekly)
\item Focused on student's questions and project progress
\end{itemize}

\textbf{Homework Philosophy:}
\begin{itemize}[noitemsep]
\item Purpose-driven mini-games (not exercises)
\item Designed for 15-30 minute daily practice
\item Optional challenges for deeper exploration
\item Midweek progress check-ins available
\end{itemize}

\textbf{Learning Foundation:}
\begin{itemize}[noitemsep]
\item Concepts introduced through game mechanics
\item Math taught visually through game physics
\item Just-in-time learning - no boring lectures!
\end{itemize}
\end{learningblock}

\section{How to Move}
\begin{itemize}[nosep, leftmargin=*]
\item \textbf{Position:} X/Y coordinates on screen
\item \textbf{Velocity:} Movement per frame
\item \textbf{Control:} Arrow keys, mouse, touch
\item \textbf{Project:} Move a character smoothly
\end{itemize}

\section{Events \& Input}
\begin{itemize}[nosep, leftmargin=*]
\item \textbf{Triggers:} Key presses, mouse clicks, collisions
\item \textbf{Handlers:} What happens when events occur
\item \textbf{Project:} Character jumps when spacebar pressed
\end{itemize}

\section{Appearance \& Animation}
\begin{itemize}[nosep, leftmargin=*]
\item \textbf{Sprites:} Loading and displaying images
\item \textbf{Frames:} Simple animation cycles
\item \textbf{Project:} Walking animation when moving
\end{itemize}

\section{Game Logic}
\begin{itemize}[nosep, leftmargin=*]
\item \textbf{Variables:} Score, health, timers
\item \textbf{Conditionals:} If-then rules
\item \textbf{Project:} Collect coins to increase score
\end{itemize}

\section{Interactions}
\begin{itemize}[nosep, leftmargin=*]
\item \textbf{Collision:} Rectangle and circle detection
\item \textbf{Triggers:} Zone-based interactions
\item \textbf{Project:} Enemy disappears when jumped on
\end{itemize}

\section{Game States}
\begin{itemize}[nosep, leftmargin=*]
\item \textbf{Screens:} Menu, Play, Game Over
\item \textbf{Transitions:} Changing between states
\item \textbf{Project:} Restart game after Game Over
\end{itemize}

\section{Final Project: Build a Game}
Create a complete game with:
\begin{itemize}[nosep, leftmargin=*]
\item \textbf{Complexity:} At least Ping Pong level
\item \textbf{Required:} 
    \begin{itemize}
    \item Player control
    \item Scoring system
    \item Win/lose conditions
    \item Visual feedback
    \end{itemize}
\item \textbf{Examples:} 
    \begin{itemize}
    \item Pong with power-ups
    \item Platformer with 3 levels
    \item Space shooter with enemy waves
    \end{itemize}
\end{itemize}

\section{Transition to Python}
\begin{itemize}[nosep, leftmargin=*]
\item \textbf{Why Python?} 
    \begin{itemize}
    \item \textbf{Saving:} Store game data permanently
    \item \textbf{Multiplayer:} Networked games
    \item \textbf{AI:} Smart enemies/opponents
    \item \textbf{Performance:} Handle complex games
    \end{itemize}
\item \textbf{First Python Project:} Rebuild your Scratch game in Python
\end{itemize}

\section{Next Steps}
\begin{itemize}[nosep, leftmargin=*]
\item \textbf{Python Track:} Continue with Python game development
\item \textbf{Advanced Topics:} 
    \begin{itemize}
    \item Physics engines
    \item Online multiplayer
    \item 3D graphics
    \item AI opponents
    \end{itemize}
\end{itemize}

\vspace{1em}
\noindent\fbox{\parbox{\textwidth}{
\textbf{Disclaimer:} After completing this core curriculum, you'll transition to the Python track where we'll recreate your Scratch projects with real code and add advanced features.
}}

\end{document}
